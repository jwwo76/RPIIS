\documentclass[10pt]{article}
\usepackage{amsmath,amsthm,amssymb}
\usepackage{mathtext}
\usepackage[T1,T2A]{fontenc}
\usepackage[utf8]{inputenc}
\usepackage[russian,english]{babel}
\usepackage[mathletters]{ucs}
\usepackage{multicol}
\usepackage{ragged2e}
\usepackage[paperwidth=216mm,paperheight=279mm,left=2.2cm,right=2.2cm, top=2.2cm,bottom=2.5cm]{geometry}
\usepackage{setspace}
\usepackage{biblatex}
\usepackage{parskip}
\setcounter{page}{78}
\title{\Huge \textbf{Adaptive User Interfaces for Intelligent
Systems: Unlocking the Potential of
Human-System Interaction}}
\author{ \small Mikhail Sadouski, Pavel Nasevich, Maksim Orlov, Alexandra Zhmyrko \\ \small  Belarusian State University \\ \small of Informatics and Radioelectronics \\ \small Minsk, Belarus \\ \small Email: sadovski@bsuir.by, fallenchromium@gmail.com, orlovmaksimkonst@gmail.com, aleksashazh@gmail.com}
\date{}
\linespread{0.84}
\setlength{\parskip}{0pt}
\usepackage{etoolbox}
\patchcmd{\thebibliography}{\section*{\refname}}{}{}{}
\begin{document}
\footnotesize
\twocolumn
\justifying

\begin{thebibliography}{99}

\setlength{\parskip}{0pt}

\bibitem{russ} V. Golenok, N. Guliakina, V. Golovko, V. Krasnoprishin, "Methodological problems of the current state of works in the field of artificial intelligence," in \textit{Otkrytie semanticheskie tekhnologii proektirovaniya intellektual’nykh sistem (Open semantic technologies for intelligent systems)}, ser. 5, V. Golenok, Ed. BSUIR, Minsk, 2021, pp. 17–24.
\bibitem{ryan2022} V. Ryan, A. Suyu, and D. Roman, "Building semantic 
knowledge graphs from (semi-) structured data: a review," \textit{Future Internet}, vol. 14, no. 5, p. 129, 2022.
\bibitem{bamgai2023} P. Bamgai, A. Sheth, and C. Henson, "From data to actionable knowledge: Big data challenges in the web of things," \textit{IEEE Intelligent Systems}, vol. 28, no. 6, pp. 6–11, 2013.
\bibitem{kellogg1986} C. Kellogg, "From data management to knowledge management," \textit{Computer}, vol. 19, no. 01, pp. 75–84, 1986.

\bibitem{shunkevich2022} D. Shunkevich, "Universal model of interpreting logical-semantic models of intelligent computer systems of a new generation," in \textit{Otkrytie semanticheskie tekhnologii proektirovaniya intellektual’nykh sistem (Open semantic technologies for intelligent systems)}, V. Golenok, Ed. BSUIR, Minsk, 2022, pp. 285–296.

\bibitem{1}Golenkov, V. V., “Graphodynamic models of parallel knowledge
processing: principles of construction, implementation, and design,”
in \textit{Open semantic technologies for designing intelligent
systems}(OSTIS-2012): \textit{Open semantic technologies for designing intelligent systems} 16-18, 2012,
Belarusian State University of Informatics and Radioelectronics;
editorial board: V. V. Golenkov (chief editor) [et al.]. Minsk,
2012, pp. 23–52.

\bibitem{2} V. Ivashenko, “General-purpose semantic representation language
and semantic space,” in \textit{Otkrytye semanticheskie tekhnologii
proektirovaniya intellektual’nykh system [Open semantic technologies
for intelligent systems]}, ser. Iss. 6. Minsk : BSUIR, 2022,
pp. 41–64.

\bibitem{3} V. V. Golenkov and N. A. Gulyakina, “Structuring the semantic
space,” in \textit{Open semantic technologies for designing intelligent
systems} (OSTIS-2014): \textit{Proceedings of the IV International Scientific
and Technical Conference}, V. V. Golenkov, Ed. Minsk:
BSUIR, 2 2014, pp. 65–78, chief editor Golenkov, V. V. [and
others].

\bibitem{4} “Principles of building mass semantic technology for component
design of intelligent systems,” in \textit{Open semantic technologies
for designing intelligent systems} (OSTIS-2011): \textit{Proceedings
of the international scientific and technical conference}, V. V.
Golenkov, Ed. Minsk: Belarusian State University of Informatics
and Radioelectronics, 2 2011, pp. 21–58, chief editor Golenkov,
V. V. [et al.].

\bibitem{5}D. Shunkevich, “Hybrid problem solvers of intelligent computer
systems of a new generation,” \textit{Otkrytye semanticheskie tekhnologii
proektirovaniya intellektual’nykh system [Open semantic technologies
for intelligent systems]}, no. 6, pp. 119–144, 2022.

\bibitem{6}“Neo4j graph Database Platform | Graph Database Management
System [Electronic resource],” April 2024. [Online]. Available:
https://neo4j.com/

\bibitem{7}T. Kahveci and A. K. Singh, “An efficient index structure for
string databases,” in VLDB, vol. 1, 2001, pp. 351–360.

\bibitem{8}M. Barsky, U. Stege, and A. Thomo, “Structures for indexing
substrings,” in \textit{Full-Text (Substring) Indexes in External Memory}.
Springer, 2012, pp. 1–15.

\bibitem{9}N. V. Zotov, “Model of process management in shared semantic
memory of intelligent systems,” in \textit{Information Technologies and
Systems} 2023 (\textit{ITS} 2023), L. Y. Shilin, Ed. Minsk: Belarusian
State University of Informatics and Radioelectronics, 11 2023, pp.
53–54, proceedings of the International Scientific Conference.

\bibitem{10}J. L. W. Kessels, “An alternative to event queues for synchronization
in monitors,” \textit{Communications of the ACM}, vol. 20, no. 7,
pp. 500–503, 1977.

\bibitem{11}C. A. R. Hoare, “Monitors: An operating system structuring
concept,” \textit{Communications of the ACM}, vol. 17, no. 10, pp. 549–
557, 1974.

\bibitem{12}A. Zagorskiy, “Principles for implementing the ecosystem of nextgeneration
intelligent computer systems,” in \textit{Otkrytye semanticheskie
tekhnologii proektirovaniya intellektual’nykh system [Open
semantic technologies for intelligent systems]}. BSUIR, Minsk,
2022, p. 347–356.

\bibitem{13}\textit{Linux system programming: talking directly to the kernel
and C library}. O’Reilly Media, Inc., 2013

\bibitem{14}“Software implementation of semantic networks processing storage
[Electronic resource],” 2024, mode of access: https://github.
com/ostis-ai/sc-machine. — Date of access: 30.03.2024.

\bibitem{15}R. Bayer, “Prefix b-trees,” ACM Transactions on Database Systems
(TODS), vol. 2, no. 1, pp. 11–26, 1977.

\bibitem{16}P. Ferragina and R. Grossi, “The string b-tree: A new data structure
for string search in external memory and its applications,”
\textit{Journal of the ACM (JACM)}, vol. 46, no. 2, pp. 236–280, 1999.

\bibitem{17}D. Belazzougui, “Fast prefix search in little space, with applications,”
in \textit{European Symposium on Algorithms}, 2010, pp. 427–
438.

\bibitem{18}M. I. Cole, \textit{Algorithmic skeletons: structured management of
parallel computation}. Pitman London, 1989.

\bibitem{19}L. Gonnord, L. Henrio, L. Morel, and G. Radanne, “A survey on
parallelism and determinism,” \textit{ACM Computing Surveys}, vol. 55,
no. 10, pp. 1–28, 2023.

\bibitem{20}N. V. Zotov, “Quantitative indicators of operations efficiency over
shared semantic memory of intelligent systems,” in \textit{Information
Technologies and Systems 2023 (ITS 2023)}, L. Y. Shilin, Ed.
Minsk: Belarusian State University of Informatics and Radioelectronics,
11 2023, pp. 51–52, proceedings of the International
Scientific Conference.

\bibitem{21}L. P. Miret, “Consistency models in modern distributed systems.
an approach to eventual consistency,” \textit{Master. MA thesis. Universitat
Politecnica de Valencia, Spain}, 2014.
\end{thebibliography}
\begin{center}
  \section*{ФОРМАЛЬНАЯ МОДЕЛЬ ОБЩЕЙ
СЕМАНТИЧЕСКОЙ ПАМЯТИ ДЛЯ
ИНТЕЛЛЕКТУАЛЬНЫХ СИСТЕМ
НОВОГО ПОКОЛЕНИЯ}

\Large
Зотов Н.В.  
\end{center}

\indent{В работе подробно рассматривается формальная модель
семантической памяти для интеллектуальных систем, струк-
тура, её элементы, соответствия между ними, правила и ал-
горитмы. Описывается реализация на основе данной модели,
приводятся количественные показатели её эффективности.}\\ 

\begin{flushright}
    \large Received 15.03.2024
\end{flushright}

\newpage

\linespread{0.9}

\maketitle

\normalsize

\textbf{\textit{Abstract}—The paper analyzes the capabilities of com\-puter
systems and the level of development of tools for
interacting with them (user interfaces). Based on the analysis,
an approach to the design of adaptive user interfaces
of intelligent systems based on the OSTIS Technology is
proposed. The semantic model of such interfaces, proposed
earlier, has been clarified and exten\-ded, the propo\-sed architecture
of such systems is given. Adaptive user interfaces
designed on the basis of the propo\-sed approach will provide
new scenarios of user interaction with computer systems.}

\textbf{\textit{Keywords}—adaptive user interface, intelligent systems,
user interface of ostis-systems}

\begin{center}
    \paragraph{I. Introduction}
\end{center}

\vspace{5mm}In the modern world, people use computer systems
of various purposes daily. A key component of such
systems that directly influences their efficiency is the user
interface, which in a broad sense is a set of tools that
provide interaction between a person and the system.
A large part of the cost of developing computer
systems is in the design, testing, and development of the
user interface [1].

A poorly designed user interface limits the potential
of the system by increasing the threshold of entry and reducing
the efficiency of interaction with users or making
some interaction scenarios impossible [2].

With the development of information society, the need
of users for computer systems capable of solving various
classes of problems, including tasks that are difficult to
formalize, has led to an increased pace of development
of computer technologies, the creation of a large number
of models, methods and tools for the design and development
of computer systems, including intelligent computer
systems with increased requirements for inter\-operability,
component compatibility and flexibility of scenarios of
interaction with the user because of their ability to
self-learning and solving complex problems, as well as
problems, in the initial data and algorithms of solution
of which there is an influence of non-factors.

However, even computer systems that we use on a daily
basis are severely limited in their functionality due to thelimited means of interaction with these systems. There
is a mismatch between the current level of development
of user interfaces and the problem-solving capabilities of
computer systems.
Each user has unique needs and the application of
adaptive user interfaces for intelligent systems becomes
essential. The ease and flexibility of dynamically changing
user interfaces based on user tasks that are not predetermined
in the design of the system becomes key and
allows for greater potential for interaction with intelligent
computer systems.
This article analyzes the capabilities of computer systems
and the level of development of tools for interacting
with them (user interfaces).
The user interface is considered in this context as the
language of communication between the system and the
user, together with the means for such communication,
emphasizing the content of the interaction more than the
specific technical aspects of the implementation of the
interaction. This analysis is based on a description of
the historical development of computer systems (and, as
a consequence, the development of the many classes of
problems that computer systems solve).
Based on the analysis, an approach to the design of
adaptive user interfaces of intelligent systems based on
the OSTIS Technology is proposed. The proposed [3]
semantic model of such interfaces is clarified and extended,
and the proposed architecture of such systems
is given. Adaptive user interfaces designed on the basis
of the proposed approach will provide new scenarios of
user interaction with computer systems.

\begin{center}
    \paragraph{II. Analysis of existing approaches to solving the
problem}
\end{center}

 \vspace{+5mm}A. \textit{Challenges of modern user interfaces}


\vspace{+5mm}The challenges of modern user interfaces that cause
users to fail to take advantage of the full potential of
computer system problem solvers can be categorized as
follows:

\begin{itemize}
    \item A mismatch between the user’s skills and means of
interacting with the system and the actual means of
interaction provided by the system. An example is
an over-complicated user interface for inexperienced
users, or a user interface that does not take into
account that the user is fully informed about the
algorithm for solving a problem and takes time
and attention away from unnecessary explanations.
This leads to the fact that the user cannot predict
the navigation through the user interface — his
cognitive load increases, the time required to solve
the problem increases [4].

\item Mismatch between the task and environment for
which the user interface was designed with the
actual task and environment of the system. This
refers to the case where the system’s problem solver
is capable of solving a broader and more general
class of problems than the system’s user interface
allows, since only in such a case can the user
interface alone be said to be the limiting factor in
the applicability of the system — [5].

\item Lack of user’s ability to make changes to the interface
for integration with other user interfaces
and/or systems. The impossibility of building an
integrated working environment (i.e. a new user
interface, qualitatively different from the multitude
of interfaces designed to solve each of the sub-tasks
of the complex task) makes it impossible to build
scenarios for automatic integration of computer system
subsystems and severely limits the user’s ability
to integrate computer systems, because without the
ability to change the user interface, the program
behavior can only be changed programmatically [6],
[7].
\end{itemize}

There are a lot of examples of functionalities limited
on the user interface side, which are nevertheless technically
realizable, but we will limit ourselves to a few
significant ones:

\begin{itemize}
    \item Lack of integration of system components: your
word processor does not prompt you to open the
last quarter’s financial report, even though you were
emailed it yesterday and are scheduled to check it
on your electronic calendar today.

\item Lack of automatic adaptation to the environment:
the online store does not offer you to make a pickup
to the branch you are closest to at the moment.

\item Lack of ability to personalize the user interface:
your fitness app does not allow you to fully customize
the start screen with health metrics that
are irrelevant to you and with workout suggestions
that do not match your goal and access to fitness
equipment.

\item Solving complex tasks: for an average user the
interface for automated solution of a multi-step task
is not standard and is always available. For example,
to process a call or a letter from a customer, enter
relevant information from the letter into the project
accounting system, assign a person with the necessary
competence to be responsible for the project
and send the customer a letter with the contact of
the person responsible for the project.

\end{itemize}

It is worth noting that for each of these examples it is
possible to find a counterexample: a computer system
that did take into account the described use case, but
it is worth recognizing that in general the described
limitations exist in the vast majority of computer systems.

\vspace{+4mm}\textit{B. Overview of the evolution of computer systems and
their user interfaces}

\vspace{+4mm}Let us consider the development paths of computer
systems and their user interfaces - the major milestones,
and what they were driven by (discoveries or user needs).
Over the last 60 years, a large number of approaches
to user interface design have been developed. The approaches
focused on different directions, such as "which
side of the interaction is primary" (the emotional state
of the user, the purpose of the system, feedback from
the user, the design of the program that implements the
algorithm for solving the problem) and "how to convey
the meaning of the objects of the user interface" —
confrontation of metaphorical and idiomatic approach
[8], [9].

At the same time, the paradigms used to implement the
user interface have evolved. For example, the transition
from a model where each interaction has an underlying
function call to perform the process and provide feedback
to an object-oriented interface where the UI elements correspond
to the properties of the system entities (reactive
approach).

Currently, the approach when the user interface is
described by a set of states and transitions in the algorithm
of problem solving is popular (i.e. the paradigm
of so-called "wizards", which leads the user step-by-step
from the need to its resolution). Each of the approaches
has a place, but a special place in the design of user
interfaces for intelligent computer systems, based on the
dynamic nature of the tasks that can be set before it, takes
the paradigm of interface design based on the problem
being solved and the algorithm for its solution. To date,
there are no comprehensive means of building adaptive
user interfaces (taking into account the environment,
the characteristics of the user and his device), while
dynamically taking into account the information about
the algorithm for solving the problem.

\vspace{4mm}\textit{C. Conclusions}

\vspace{4mm}Based on the representation of the user interface described
in I, we can conclude that the problems described
in II-A are the result of a methodological error in the
design of interfaces. At the moment, the factors affecting
how the user interface should look like are taken into

\end{document}